%
% \iffalse
%<*driver>
\ProvidesFile{tikzlib.dtx}
%</driver>
%<tikzlib>\ProvidesFile{tikzlibraryMATH-UA-123-fall22.code.tex}
%<*tikzlib>
  [2022/09/11 v0.2c TikZ Library for MATH-UA 123, Fall 2022]
%</tikzlib>
%<*driver>
\documentclass{ltxdoc}
\EnableCrossrefs
\CodelineIndex
\begin{document}
  \DocInput{tikzlib.dtx}
\end{document}
%</driver>
% \fi
%
% \GetFileInfo{tikzlib.dtx}
% \title{A TikZ library for Calculus III, Fall 2022}
% \date{\fileversion \qquad \filedate}
% \maketitle
%
% \begin{abstract}
% This is the documentation of the Module One example.
% \end{abstract}
%
% \section{Introduction}
%
% This is where you would explain the package to a user.
%
% \StopEventually{}
%
% \section{Implementation}
%
%    \begin{macrocode}
%<*tikzlib>
%    \end{macrocode}
%
% This library builds on a base library.
%    \begin{macrocode}
\usetikzlibrary{MATH-UA-123}
%    \end{macrocode}
%
% Load the 2022 NYU colors and map the generic names onto them.
%    \begin{macrocode}
\usepackage{xcolor-nyu22}
\colorlet{primary}{NyuViolet}
\colorlet{secondary}{LightViolet1}
\colorlet{tertiary}{UltraViolet}
\colorlet{quartenary}{MediumViolet2}
\usetikzlibrary{backgrounds}
%    \end{macrocode}
%
% This allows for setting the fill color of the background rectangle of a TikZ
% picture. It's useful because some of the block backgrounds aren't white
% anymore.
%    \begin{macrocode}%
\tikzset{background fill/.style={
    background rectangle/.style={fill=#1},
    show background rectangle,
    },
}
%    \end{macrocode}
%
% That's all, folks!
%    \begin{macrocode}
%</tikzlib>
%    \end{macrocode}
%
% \Finale
%
